\documentclass[11pt,spanish]{article}
\usepackage[utf8]{inputenc}
\usepackage{ragged2e}
\usepackage{url}
\usepackage{setspace}
\usepackage{amsmath}
\onehalfspacing
\usepackage{graphicx}
\usepackage[font=small,labelfont=it]{caption}
\usepackage[margin=2.5cm]{geometry}
\usepackage{booktabs}
\usepackage[export]{adjustbox} 
\graphicspath{ {images/} }

\title{EAFIT UNIVERSITY\\
COMPUTER SCIENCE DEPARTMENT\\
CLASS METHODS IMPLEMENTATION}
\author{
  Gutiérrez Gómez, Mateo\\
  \texttt{mgutie22@eafit.edu.co}
    \and
  Marulanda Cifuentes, Mateo\\
  \texttt{mmarulandc@eafit.edu.co}
    \and
  Narvaez Pulgarin, Yashua\\
  \texttt{yanarvaezp@eafit.edu.co}
}

\date{November 11, 2020}

\begin{document}

\maketitle

\section{Introduction}

For every method an matrix ,we are going to use the following formulas and test matrix, also we are going to use a $ Tol = 10^-7$ and $ N = 100 $

\section{Used data:}
\begin{center}
Matrix and vectors:\\

$A= \begin{pmatrix}
4 & -1 & 0 & 3\\
1 & 15.5 & 3 & 8\\
0 & -1.3 & -4 & 1.1\\
14 & 5 & -2 & 30
\end{pmatrix}$ 
,  
$b= \begin{pmatrix}
1\\
1\\
1\\
1
\end{pmatrix}$
,
$x0= \begin{pmatrix}
0\\
0\\
0\\
0
\end{pmatrix}$
\end{center}
\begin{table}
Tables:\\
\centering
Table =
\begin{tabular}{|l|l|l|l|l|} 
\hline
X & -1 & 0 & 3 & 4  \\ 
\hline
Y & 15.5 & 3 & 8 & 1  \\
\hline
\end{tabular}
\end{table}
\pagebreak

\section{LU Factorization Using Gauss}
Input values: A,b
\begin{center}
    \includegraphics [scale=0.7]{lu_gauss.png}
\end{center}

\pagebreak
\section{LU Factorization Using Partial Pivoting}
Input values: A,b
\begin{center}
    \includegraphics [scale=0.7]{lu_factorization_partial_pivoting.png}
\end{center}

\section{Crout}
Input values: A,b
\begin{center}
    \includegraphics [scale=0.7]{crout.png}
\end{center}
\pagebreak

\section{Doolittle}
Input values: A,b
\begin{center}
    \includegraphics [scale=0.7]{doolittle.png}
\end{center}

\section{Cholesky}
Input values: A,b
\begin{center}
    \includegraphics [scale=0.7]{cholesky.png}
\end{center}

\pagebreak
\section{Jacobi}
Input values: A,b, $x_0$ , Tol y N \
\begin{center}
    \includegraphics [scale=0.6]{jacobi.png}
\end{center}

\pagebreak
\section{Gauss-Seidel}
Input values: A,b, $x_0$ , Tol y N\
\begin{center}
    \includegraphics [scale=0.5]{gauss-seidel.png}
\end{center}

\pagebreak
\section{SOR Methods}
Input values: A,b, $x_0$ , Tol, N, W=1.5\
\begin{center}
    \includegraphics [scale=0.7]{jacobi_sor.png}
    \includegraphics [scale=0.7]{gauss_seidel_sor.png}
\end{center}

\section{Vandermonde}
Input values: Table
\begin{center}
    \includegraphics [scale=0.7]{vandermonde.png}
\end{center}

\pagebreak
\section{Newton Interpolation}
Input values: Table
\begin{center}
    \includegraphics [scale=0.5]{newton-interpolation.png}
\end{center}

\section{Lagrange}
Input values: Table
\begin{center}
    \includegraphics [width=\textwidth]{lagrange.png}
\end{center}


\section{Linear Spline}
Input values: Table
\begin{center}
    \includegraphics [width=\textwidth]{lineal-spline.png}
\end{center}

\section{Quadratic Spline}
Input values: Table
\begin{center}
    \includegraphics [width=\textwidth]{quadratic-spline.png}
\end{center}

\pagebreak
\section{Cubic Spline}
Input values: Table
\begin{center}
    \includegraphics [width=\textwidth]{cubic-spline.png}
\end{center}

\end{document}